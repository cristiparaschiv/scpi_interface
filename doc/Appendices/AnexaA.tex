% Appendix A

\chapter{Sursa fi\c{s}ierului Lex}
\label{AnexaA}
\lhead{Anexa A. \emph{Sursa fi\c{s}ierului Lex}}

\begin{lstlisting}
%{
#include <stdio.h>
#include <string.h>
#include "y.tab.h"
char fr, sr;
%}

%%
SOURce:VOLTage		return VOLTAGE;

SOURce:AM:SOURce	return AM_SOURCE;
SOURce:AM:DEPTh		return AM_DEPTH;
SOURce:AM:INTernal:FREQuency	return AM_FREQ;
SOURce:APPLy:SIN	return AM_FREQ1;

SOURce:FM:SOURce	return FM_SOURCE;
SOURce:FM:INTernal:FREQuency	return FM_FREQ;
SOURce:FM:DEViation	return FM_PEAK;
SOURce:FREQuency	return FREQ;
SOURce:FM:STATe		return FM_START;

SOURce:SWEep:STATe	return SW_START;
SOURce:FREQuency:STARt	return SW_FR_START;
SOURce:FREQuency:STOP	return SW_FR_STOP;
SOURce:SWEep:SPACing	return SW_LIN_LOG;
SOURce:SWEep:DIRection	return SW_UP_DOWN;
SOURce:SWEep:TIME	return SW_TIME;

SOURce:PM:SOURce	return PM_SOURCE;
SOURce:PM:INTernal:FREQuency	return PM_FREQ;
SOURce:PM:DEViation	return PM_DEVIATION;
SOURce:PM:STATe		return PM_START;

SOURce:FREQuency:FSKey:SOURce	return FSK_SOURCE;
SOURce:FREQuency:FSKey:MARK	return FSK_MARK;
SOURce:FREQuency:FSKey:SPACe	return FSK_SPACE;
SOURce:FSKey:STATe	return FSK_START;

INTernal	{yylval=yytext;return INTERNAL;}
EXTernal	{yylval=yytext;return EXTERNAL;}
SOURce:AM:STATe	return START;
ON	return OK;
LINear	return LINEAR;
LOGarithmic	return LOGARITHMIC;
UP	return SW_UP;
DOWN	return SW_DOWN;
SINusoid	return SINUSOID;
FUNCtion	return FUNCTION;
TRIGger		return TRIGGER;
NOISe		return NOISE;
TRIangle	return TRIANGLE;

[0-9]+			{yylval=atoi(yytext);return NUMBER;}
\n
[ \t]+


%%
\end{lstlisting}
