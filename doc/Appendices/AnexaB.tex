% Appendix A

\chapter{Sursa fi\c{s}ierului Yacc}
\label{AnexaB}
\lhead{Anexa B. \emph{Sursa fi\c{s}ierului Yacc}}

\begin{lstlisting}
%{
#include <stdio.h>
#include <string.h>
#include <stdlib.h>
#include <windows.h>
#include <tchar.h>
HANDLE hPort;
char DataBuffer[];
char* surs;
char* spac;
char* dir;
int freq1,freq,depth,peak,start_freq,stop_freq,time,deviat,level,n1,n2,len;
%}

%token AM_SOURCE AM_DEPTH AM_FREQ AM_FREQ1 INTERNAL EXTERNAL NUMBER START VIRGULA VOLTAGE
%token FM_SOURCE FM_FREQ FM_PEAK FREQ FM_START OK
%token SW_START SW_FR_START SW_FR_STOP SW_LIN_LOG SW_UP_DOWN SW_TIME LINEAR LOGARITHMIC SW_UP SW_DOWN
%token PM_SOURCE PM_FREQ PM_DEVIATION PM_START
%token SINUSOID FUNCTION TRIGGER NOISE TRIANGLE
%token FSK_SOURCE FSK_MARK FSK_SPACE FSK_START
%%

commands: /* empty */
	| commands command
	;
numbers:
	numbers NUMBER	
	|
	NUMBER
	;
command:
	comm numbers
	|
	comm sursa
	|
	comm spacing
	|
	comm direct
	|
	comm
	;
sursa:
	INTERNAL	{surs="1";printf(" intern\n");}
	|
	EXTERNAL	{surs="2";printf(" extern\n");}
	;
spacing:
	LINEAR		{spac="1";printf(" linear\n");}
	|
	LOGARITHMIC	{spac="0";printf(" logarithmic\n");}
	;
direct:
	SW_UP	{dir="0";printf(" up\n");}
	|
	SW_DOWN	{dir="1";printf(" down\n");}
	;
comm:
	AM_FREQ1 {printf("  >>carrier freq:");}
	| VOLTAGE NUMBER {printf("  >>level: %i\n", $2);level=$2;}
	| AM_SOURCE	{printf("  >>source:");}
	| AM_DEPTH NUMBER	{printf("  >>percentage modulation: %i\n",$2);depth=$2;}
	| AM_FREQ NUMBER	{printf("  >>modulating freq: %i\n", $2);freq=$2;}
	| START	OK	{
				if(surs=="1")
					{
						printf("M1 %s F1 %iX F2 %iZ F3 %iX F4 %iY\n", surs, freq, depth, freq1, level);
						sprintf(DataBuffer,"M1 %s F1 %iX F2 %iZ F3 %iX F4 %iY", surs, freq, depth, freq1, level);
						len = strlen(DataBuffer);
						WriteString(DataBuffer, len);
					}
				else
					{
						printf("M1 %s F1 0Z F2 %iX F3 %iY\n", surs, freq1,level);
						sprintf(DataBuffer,"M1 %s F1 0Z F2 %iX F3 %iY", surs, freq1,level);
						len = strlen(DataBuffer);
						WriteString(DataBuffer, len);
					}
			}
	| FM_SOURCE	{printf("  >>source:");}
	| FM_FREQ NUMBER	{printf("  >>modulating freq: %i\n",$2);freq=$2;}
	| FM_PEAK NUMBER	{printf("  >>peak freq: %i\n",$2);peak=$2;}
	| FREQ NUMBER	{printf("  >>carrier freq: %i\n",$2);freq1=$2;}
	| FM_START OK	{
		if(surs=="1")
			{
				printf("M2 %s F1 %iX F2 %iX F3 %iX F4 %iY\n", surs, freq, peak, freq1, level);
				sprintf(DataBuffer,"M2 %s F1 %iX F2 %iX F3 %iX F4 %iY", surs, freq, peak, freq1, level);
				len = strlen(DataBuffer);
				WriteString(DataBuffer, len);
			}
		else
			{
				printf("M2 %s F1 %iX F2 %iX F3 %iY\n", surs, peak, freq1, level);
				sprintf(DataBuffer,"M2 %s F1 %iX F2 %iX F3 %iY", surs, peak, freq1, level);
				len = strlen(DataBuffer);
				WriteString(DataBuffer, len);
			}
	}
	| SW_FR_START NUMBER	{printf("  >>start freq: %i\n", $2);start_freq=$2;}
	| SW_FR_STOP NUMBER	{printf("  >>stop freq: %i\n", $2);stop_freq=$2;}
	| SW_LIN_LOG	{printf("  >>lin\log sweep:");}
	| SW_UP_DOWN	{printf("  >>direction:");}
	| SW_TIME NUMBER	{printf("  >>sweep time: %i\n", $2);time=$2;}
	| SW_START OK	{
				printf("M4 F1 %iX F2 %iX F3 %s F4 %i F5 %s F6 %iX F7 %iY\n", start_freq, stop_freq, spac, surs, dir, time, level);
				sprintf(DataBuffer,"M4 F1 %iX F2 %iX F3 %s F4 %i F5 %s F6 %iX F7 %iY", start_freq, stop_freq, spac, surs, dir, time, level);
				len = strlen(DataBuffer);
				WriteString(DataBuffer, len);
			}
	| PM_SOURCE	{printf("  >>source:");}
	| PM_FREQ NUMBER	{printf("  >>modulating freq: %i\n", $2);freq=$2;}
	| PM_DEVIATION NUMBER	{printf("  >>peak phase deviation: %i\n", $2);deviat=$2;}
	| PM_START OK	{
				if(surs=="1")
				{			
					printf("M3 %s F1 %iX F2 %iX F3 %iX F4 %iY\n", surs, freq, deviat, freq1, level);
					sprintf(DataBuffer,"M3 %s F1 %iX F2 %iX F3 %iX F4 %iY", surs, freq, deviat, freq1, level);
					len = strlen(DataBuffer);
					WriteString(DataBuffer, len);
				}
				else
				{			
					printf("M3 %s F1 %i F2 %iX F3 %iY\n", deviat, freq1, level);
					sprintf(DataBuffer,"M3 %s F1 %i F2 %iX F3 %iY", deviat, freq1, level);
					len = strlen(DataBuffer);
					WriteString(DataBuffer, len);
				}
	}
	| FUNCTION SINUSOID 	{
					printf("B F1 %iX F2 %iY\n", freq1, level);
					sprintf(DataBuffer,"B F1 %iX F2 %iY", freq1, level);
					len = strlen(DataBuffer);
					WriteString(DataBuffer, len);
				}
	| FUNCTION NOISE	{
					printf("V F1 3 F2 %i F3 %iX F4 %iY\n", surs, freq1, level);
					sprintf(DataBuffer, "V F1 3 F2 %i F3 %iX F4 %iY", surs, freq1, level);
					len = strlen(DataBuffer);
					WriteString(DataBuffer, len);
				}
	| FUNCTION TRIANGLE	{
					printf("V F1 2 F2 %i F3 %iX F4 %iY\n", surs, freq, level);
					sprintf(DataBuffer, "V F1 2 F2 %i F3 %iX F4 %iY", surs, freq, level);
					len = strlen(DataBuffer);
					WriteString(DataBuffer, len);
				}
	| TRIGGER {printf("  >>trigger source:");}
	| FSK_MARK NUMBER {printf("  >>mark freq: %i\n", $2);n1=$2;}
	| FSK_SPACE NUMBER {printf("  >>space freq: %i\n", $2);n2=$2;}
	| FSK_SOURCE {printf("  >>source:");}
	| FSK_START OK {
		if(surs=="1")
			{
				printf("M5 %s F1 %iX F2 %iX F3 %iX %iY\n", surs, freq1, n1, n2, level);
				sprintf(DataBuffer,"M5 %s F1 %iX F2 %iX F3 %iX %iY", surs, freq1, n1, n2, level);
				len = strlen(DataBuffer);
				WriteString(DataBuffer, len);
			}
		else
			{
				printf("M5 %s F1 %iX F2 %iX F3 %iY\n", surs,n1,n2,level); 
				sprintf(DataBuffer,"M5 %s F1 %iX F2 %iX F3 %iY", surs,n1,n2,level);
				len = strlen(DataBuffer);
				WriteString(DataBuffer, len);
			}
	}	
	;
%%

//--------------------
HANDLE ConfigureSerialPort(LPCSTR  lpszPortName)
{
    HANDLE hComm = NULL;
    DWORD dwError;
    DCB PortDCB;
    COMMTIMEOUTS CommTimeouts;
    hComm = CreateFile (lpszPortName,
        GENERIC_READ | GENERIC_WRITE,
        0,
        NULL,
        OPEN_EXISTING,
        0,
        NULL);

    PortDCB.DCBlength = sizeof (DCB);
    GetCommState (hComm, &PortDCB);
    PortDCB.BaudRate = 9600;
    PortDCB.fBinary = TRUE;
    PortDCB.fParity = TRUE;
    PortDCB.fOutxCtsFlow = FALSE;
    PortDCB.fOutxDsrFlow = FALSE;
    PortDCB.fDtrControl = DTR_CONTROL_ENABLE;
    PortDCB.fDsrSensitivity = FALSE;
    PortDCB.fTXContinueOnXoff = TRUE;
    PortDCB.fOutX = FALSE;
    PortDCB.fInX = FALSE;
    PortDCB.fErrorChar = FALSE;
    PortDCB.fNull = FALSE;
    PortDCB.fRtsControl = RTS_CONTROL_ENABLE;
    PortDCB.fAbortOnError = FALSE;
    PortDCB.ByteSize = 8;
    PortDCB.Parity = NOPARITY;
    PortDCB.StopBits = ONESTOPBIT;

    if (!SetCommState (hComm, &PortDCB))
    {
        printf("Could not configure serial port\n");
        return NULL;
    }

    GetCommTimeouts (hComm, &CommTimeouts);
    CommTimeouts.ReadIntervalTimeout = MAXDWORD;
    CommTimeouts.ReadTotalTimeoutMultiplier = 0;
    CommTimeouts.ReadTotalTimeoutConstant = 0;
    CommTimeouts.WriteTotalTimeoutMultiplier = 0;
    CommTimeouts.WriteTotalTimeoutConstant = 0;
    if (!SetCommTimeouts (hComm, &CommTimeouts))
    {
        printf("Could not set timeouts\n");
        return NULL;
    }
    return hComm;
}

void ClosePort()
{
    CloseHandle(hPort);
    return;
}

BOOL WriteByte(BYTE bybyte)
{
    DWORD iBytesWritten=0;
    DWORD iBytesToRead = 1;
    if(WriteFile(hPort,(LPCVOID) 
        &bybyte,iBytesToRead,&iBytesWritten,NULL)==0)
        return FALSE;
    else return TRUE;
}

BOOL WriteString(const void *instring, int length)
{
    int index;
    BYTE *inbyte = (BYTE *) instring;
    for(index = 0; index< length; ++index)
    {
        if (WriteByte(inbyte[index]) == FALSE)
            return FALSE;
    }
    return TRUE;
}
//---------------------

int yywrap()
{
	return 1;
}

void prompt()
{
	printf(">");
}

main()
{
	//prompt();
    hPort = ConfigureSerialPort("COM1");
    if(hPort == NULL)
    {
        printf("Com port configuration failed\n");
        return -1;
    }
	yyparse();
	ClosePort();
}

  int yyerror(void)
  {
      printf("Error\n");
      exit(1);
  }


\end{lstlisting}
